\documentclass[12pt,a4paper]{article}
%\usepackage[allfiguresdraft]{draftfigure}
%\newcommand{\pdfextension}{pdf}
%\newcommand{\pngextension}{png}
\usepackage{cite}
\usepackage{natbib}
\usepackage{breakcites}
\usepackage[normalem]{ulem}
\usepackage{comment}
\usepackage[font=footnotesize,labelfont=bf]{caption}
 \usepackage[dvipsnames]{xcolor}
\let\rho=\varrho

\def\fref#1{Fig.~\ref{#1}}
\def\tabref#1{Table~\ref{#1}}
\def\cref#1{Condition~\ref{#1}}
\def\Cref#1{Corollary~\ref{#1}}
\def\eref#1{(\ref{#1})}
\def\sref#1{\textsection\ref{#1}}
\def\lref#1{Lemma~\ref{#1}}
\def\rref#1{Remark~\ref{#1}}
\def\tref#1{Theorem~\ref{#1}}
\def\dref#1{Definition~\ref{#1}}
\def\pref#1{Proposition~\ref{#1}}
\def\aref#1{Assumption~\ref{#1}}
\def\qref#1{Subsection~\ref{#1}}

\def\dref#1{Definition~\ref{#1}}
\def\myundefined#1{\special{ps: 1 0 0 setrgbcolor}{what is #1?}%
	\special{ps: 0 0 0 setrgbcolor}}
\newenvironment{myitem}
{\begin{itemize}
  \setlength{\itemsep}{1pt}
  \setlength{\parskip}{0pt}
  \setlength{\parsep}{0pt}}
{\end{itemize}}

\newenvironment{myenum}
{\begin{enumerate}
  \setlength{\itemsep}{1pt}
  \setlength{\parskip}{0pt}
  \setlength{\parsep}{0pt}}
{\end{enumerate}}

%\usepackage{sub_JP}
%\usepackage{fancyhdr}
%\pagestyle{fancy}
\usepackage{amsmath}
\usepackage{amsfonts}
\usepackage{graphicx}
%\usepackage{makeidx}
\usepackage{times}
\usepackage{amsthm}
\usepackage{ amssymb }
\usepackage{color}
\usepackage{mhequ}
\usepackage{dsfont}
\usepackage[scanall]{psfrag}
\usepackage[margin=2.5cm]{geometry}
\usepackage{color}
\usepackage{url}
\usepackage{lipsum}
%\usepackage{authblk}
\usepackage{subfig}
\usepackage{mhequ}

\usepackage{tikz}
\usepackage[graphics, active, tightpage]{}

%\usepackage[expansion=true]{microtype}

%\captionsetup[figure]{margin=2cm,font=footnotesize,labelfont=bf,labelsep=endash,textfont=rm}\captionsetup[subfigure]{margin=0pt}


\def\thecomma{\ifx,\thenewxt \else\ifx;\thenext \else\ifx.\thenext
	\else\ifx!\thenext \else\ifx:\thenext\else\ifx)\thenext \else \
	\fi\fi\fi\fi\fi\fi}
\def\condblank{\futurelet\thenext\thecomma}
\def\ie{{\it i.e.,}\condblank}
\def\eg{{\it e.g.,}\condblank}

\numberwithin{equation}{section}



\newtheorem*{oneshot}{Hypothesis~\ref{h:assumption}}
\newtheorem{theorem}{Theorem}[section]
\newtheorem{lemma}[theorem]{Lemma}
\newtheorem{proposition}[theorem]{Proposition}
\newtheorem{definition}[theorem]{Definition}
\newtheorem{notation}[theorem]{Notation}
\newtheorem{observation}[theorem]{Observation}
\newtheorem{assumption}[theorem]{Assumption}
\newtheorem{hypothesis}[theorem]{Hypothesis}
\newtheorem{conjecture}[theorem]{Conjecture}
\newtheorem{example}[theorem]{Example}
\newtheorem{property}[theorem]{Property}
\newtheorem{condition}[theorem]{Condition}
\newtheorem{corollary}[theorem]{Corollary}
\theoremstyle{definition} %remarks in rm
\newtheorem{remark}[theorem]{Remark}
\def\HALF{{\textstyle\frac{1}{2}}}
\def\TWOTHIRDS{{\textstyle\frac{2}{3}}}
\def\FOURTHIRDS{{\textstyle\frac{4}{3}}}
%\bibliographystyle{JPE}
%\usepackage{cite}

\usepackage{stmaryrd}
\def\scomm#1#2{\ensuremath{\bigr\llbracket #1:#2\bigr\rrbracket}}

\newcommand{\meq}[1]{\ensuremath{\stackrel{\scriptscriptstyle#1}{\scriptstyle
			\sim}}}


\newcommand{\dd}{\mathrm{d}}
\newcommand{\dt}{\,\dd t}
\newcommand{\mpart}[2]{\frac{\partial #1}{\partial #2}}
\newcommand{\avg}[1]{\left\langle #1\right\rangle}

\newcommand{\bigoh}[1]{\hat {\mathcal O}(p_2^{#1})}
\newcommand{\bigohneg}[1]{\hat {\mathcal O}\!\left({p_2^{-#1}}\right)}

\newcommand{\jp }[1]{{\color{magenta}jp*****:  #1}}
\newcommand{\noe }[1]{{\color{red}noe:  #1}}
\newcommand{\christophe }[1]{{\color{red}christophe: #1}}
\def\argcdot{{\,\cdot\,}}


\newcommand{\cA}{{\ensuremath{\mathcal A}} }
\newcommand{\cF}{{\ensuremath{\mathcal F}} }
\newcommand{\cP}{{\ensuremath{\mathcal P}} }
\newcommand{\cE}{{\ensuremath{\mathcal E}} }
\newcommand{\cH}{{\ensuremath{\mathcal H}} }
\newcommand{\cC}{{\ensuremath{\mathcal C}} }
\newcommand{\cN}{{\ensuremath{\mathcal N}} }
\newcommand{\cL}{{\ensuremath{\mathcal L}} }
\newcommand{\cT}{{\ensuremath{\mathcal T}} }
\newcommand{\cD}{{\ensuremath{\mathcal D}} }
\newcommand{\cU}{{\ensuremath{\mathcal U}} }
\newcommand{\cV}{{\ensuremath{\mathcal V}} }
\newcommand{\cS}{{\ensuremath{\mathcal S}} }
\newcommand{\cY}{{\ensuremath{\mathcal Y}} }


%%%%%%%%%%%%%%%%%%%%%%%%%%%%%%%%%%%%%%%%%%%%%%%%%%%%%%%%%%%%%%%%%%%%%%%%%%%%%%
%%%%%%%%%%%% Blackboard bolds
%%%%%%%%%%%%%%%%%%%%%%%%%%%%%%%%%%%%%%%%%%%%%%%%%%%%%%%%%%%%%%%%%%%%%%%%%%%%%%

\newcommand{\bbA}{{\ensuremath{\mathbb A}} }
\newcommand{\bbB}{{\ensuremath{\mathbb B}} }
\newcommand{\bbC}{{\ensuremath{\mathbb C}} }
\newcommand{\bbD}{{\ensuremath{\mathbb D}} }
\newcommand{\bbE}{{\ensuremath{\mathbb E}} }
\newcommand{\bbF}{{\ensuremath{\mathbb F}} }
\newcommand{\bbG}{{\ensuremath{\mathbb G}} }
\newcommand{\bbH}{{\ensuremath{\mathbb H}} }
\newcommand{\bbI}{{\ensuremath{\mathbb I}} }
\newcommand{\bbJ}{{\ensuremath{\mathbb J}} }
\newcommand{\bbK}{{\ensuremath{\mathbb K}} }
\newcommand{\bbL}{{\ensuremath{\mathbb L}} }
\newcommand{\bbM}{{\ensuremath{\mathbb M}} }
\newcommand{\bbN}{{\ensuremath{\mathbb N}} }
\newcommand{\bbO}{{\ensuremath{\mathbb O}} }
\newcommand{\bbP}{{\ensuremath{\mathbb P}} }
\newcommand{\bbQ}{{\ensuremath{\mathbb Q}} }
\newcommand{\bbR}{{\ensuremath{\mathbb R}} }
\newcommand{\bbS}{{\ensuremath{\mathbb S}} }
\newcommand{\bbT}{{\ensuremath{\mathbb T}} }
\newcommand{\bbU}{{\ensuremath{\mathbb U}} }
\newcommand{\bbV}{{\ensuremath{\mathbb V}} }
\newcommand{\bbW}{{\ensuremath{\mathbb W}} }
\newcommand{\bbX}{{\ensuremath{\mathbb X}} }
\newcommand{\bbY}{{\ensuremath{\mathbb Y}} }
\newcommand{\bbZ}{{\ensuremath{\mathbb Z}} }


%%%%%%%%%%%%%%%%%%%%%%%%%%%%%%%%%%%%%%%%%%%%%%%%%%%%%%%%%%%%%%%%%%%%%%%%%%%%%%
%%%%%%%%%%%% Greek letters
%%%%%%%%%%%%%%%%%%%%%%%%%%%%%%%%%%%%%%%%%%%%%%%%%%%%%%%%%%%%%%%%%%%%%%%%%%%%%%

\newcommand{\ga}{\alpha}
\newcommand{\gb}{\beta}
\newcommand{\gga}{\gamma}            % \gg already exists...
\newcommand{\gd}{\delta}
\newcommand{\gep}{\varepsilon}       % \ge already exists...
\newcommand{\gp}{\varphi}
\newcommand{\gr}{\rho}
\newcommand{\gvr}{\varrho}
\newcommand{\gz}{\zeta}
\newcommand{\gG}{\Gamma}
\newcommand{\gP}{\Phi}
\newcommand{\gD}{\Delta}
\newcommand{\gk}{\kappa}
\newcommand{\go}{\omega}
\newcommand{\gto}{{\tilde\omega}}
\newcommand{\gO}{\Omega}
\newcommand{\gl}{\lambda}
\newcommand{\gL}{\Lambda}
\newcommand{\gs}{\sigma}
\newcommand{\gS}{\Sigma}
\newcommand{\gt}{\vartheta}
\let\kappa=\varkappa
\let\phi=\varphi
\def\CC{{\mathcal C}}
\def\Rsch{r_{\rm sch}}
\def\d{{\rm d}} 
\def\KK{{\mathcal K}}
\def\OO{{\mathcal O}}
\def\integer{{\mathbb Z}}
\def\real{{\mathbb R}}
\newcommand{\ind}{\mathbf{1}}
\def\p2t2{{\tilde p_2^{\,2}}}

\def\fhc#1{\textcolor{violet}{#1}}
\definecolor{bittersweet}{rgb}{1.0, 0.44, 0.37}
\newcommand{\fhn}[1]{{\textcolor{bittersweet}{\sf[#1]}}}
\newcommand{\edit}[2] {\textcolor{black}{\sout{#1}} {\textcolor{violet}{#2}} }
\newcommand{\HZ}[1]{\textcolor{ForestGreen}{HZ:#1}}
\def\JP#1{\textcolor{blue}{JPE:#1}}
\let\epsilon=\varepsilon
\usepackage{authblk}

%%%%%%%
%% Farbod Hassani
%%%%%%%
\newcommand{\HH}{\mathcal H}
\def\be{\begin{equation}}
\def\ee{\end{equation}}
\def\const{\text{const.}}
\def\citep#1{[#1]}
\begin{document}
\section{$\alpha =0$}
Consider, again
\begin{equ}\label{eq:azero}
  u_{tt}=u_{xx}~,
\end{equ}
with initial condition $u(x,0)=0$ and $u_t(x,0)=g(x)$,
with $g(x)$ having compact support, $g'(0)=0$, $g''(0)>0$. I typically
think of $g(x)=-((1+\cos(x))/2)^n$, restricted to $[-\pi,\pi]$, which
is differentiable when e.g.~$n=2$.

The solution of \eref{eq:azero} is
\begin{equ}
  u(x,t)=\HALF \int_{x-t}^{x+t} g(\xi) \d \xi~.
\end{equ}
This implies that
\begin{equ}
  u_{xx}(x,t)=\HALF( g'(x+t)-g'(x-t) )~.
\end{equ}
Note now that if (for our example) $t>\pi/2$, the two terms above have
distinct supports and for $x<0$,
\begin{equ}
   u_{xx}(x,t)=\HALF g'(x+t)~.
\end{equ}
Since $g'(x)$ (inside its support) only vanishes for $x=0$,
we conclude that when $x$ is not too far from $t$ then
\begin{equ}\label{eq:moving}
  u_{xx}(x,t)=0  \text { if and only if } x\pm t=0~.
\end{equ}
Intuitively, this means that the root of $u_{xx}$ moves away from 0
with exactly the speed of the wave.


\section{$\alpha >0$}
In this case, the idea is that if $\alpha $ is small, then the roots
of $u_{xx}$ move \emph{away} from the origin, while for $\alpha $
large enough, they move \emph{towards} the origin, leading to local blowup.

Note that $u_{xx}(x_0,t)=0$ means that $|u_x(x_0,t)|$ is extremal.
Therefore, the extrema of $u_x(\cdot,t)$ (near 0), are at the positions
where the quantity $u_{xx}$ vanishes. One should use now the formula
\begin{equ}\label{eq:peter}
  u(x,t)=u_0(x,t) +\alpha \int_0^t \d\tau \int_{x-t+\tau }^{x+t-\tau}  \d\xi\,
  (u_x(\xi,\tau))^2\equiv u_0(x,t)+\alpha s(x,t)~,
\end{equ}
to estimate how the $\alpha $-dependent term moves the roots of
$u_{xx}$.


Consider the implicit equation
\begin{equ}
  u_{xx}(z(t),t)=0~.
\end{equ}
By what we have said before, when $\alpha =0$, then $z_0(t)=-t$ is a
possible solution (when $t>\pi/2$). We now want to show that when
$\alpha >0$ then $z_\alpha (t)>z_0(t)=-t$. (Here, the index $\alpha $ is not
a derivative, just a sign for the $\alpha $-dependence.)
Write
\begin{equa}
  s(x,t) = \int_0^t \d \tau \int_{x-t+\tau }^{x+t-\tau } \d \xi
  f(\xi,\tau )~.
\end{equa}

The derivatives of $s$ are then
\begin{equa}
  s_x&=\int_0^t \d\tau \,\left(f(x+t-\tau ,\tau )-f(x-t+\tau ,\tau) \right)~,\\
  s_{xx}&=\int_0^t \d\tau \,\left(f_x(x+t-\tau ,\tau )-f_x(x-t+\tau ,\tau) \right)~,\\
  s_{xxx}&=\int_0^t \d\tau \,\left(f_{xx}(x+t-\tau ,\tau )-f_{xx}(x-t+\tau ,\tau) \right)~,\\
  s_{xxt}&=\int_0^t \d\tau \,\left(f_{xx}(x+t-\tau ,\tau )+f_{xx}(x-t+\tau ,\tau) \right)~.\\
\end{equa}
Since $u_{0,xx}(x,t)=\HALF (g'(x+t)-g'(x-t))$,
we also have
\begin{equa}
  u_{0,xxx}&=\HALF(g''(x+t)-g''(x-t))~,\\
  u_{0,xxt}&=\HALF(g''(x+t)+g''(x-t))~.\\
\end{equa}
Therefore,
\begin{equ}
  z_\alpha '(t)=-\frac{u_{xxt}(z(t),t)}{u_{xxx}(z(t),t)}=
 - \frac{u_{0,xxt}+\alpha s_{xxt}}{u_{0,xxx}+\alpha s_{xxx}}~.
\end{equ}
We cannow restrict to negative $x$, and then we get
\begin{equ}
  z_\alpha '(t)=
 - \frac{\HALF g''(x+t)+\alpha s_{xxt}}{\HALF g''(x+t)+\alpha s_{xxx}}~.
\end{equ}
Clearly, for $\alpha =0$ we find $z_0'(t)=-1$.


Consider now ``our'' case when $f=(u_x)^2$.
Then, $f_x= 2 u_x u_{xx} $ and
\begin{equa}
  f_{xx}=2(u_{xx})^2+2 u_{xxx} u_x~.
\end{equa}

\section{Perturbation theory}

Let us replace \eref{eq:peter} by the first order approximation:
\begin{equ}
 u_0(x,t)+\alpha \int_0^t \d\tau \int_{x-t+\tau }^{x+t-\tau }
 (u_{0,x}(\xi,\tau ))^2~.
\end{equ}
Then the implicit equation for $z$  is
\begin{equ}
  u_{0xx}(z(t),t)+\left .\alpha \partial_x^2\int_0^t \d\tau \int_{x-t+\tau }^{x+t-\tau }
  (u_{0x}(\xi,\tau ))^2\right|_{x=z(t)}=0~.
\end{equ}
Let $X=\int_0^t \d\tau \int_{x-t+\tau }^{x+t-\tau }
  (u_{0x}(\xi,\tau ))^2$.  
Looking only on the negative ($x$) side, we have $u_{0x}(x,t)=\HALF g(x+t)$ and therefore
this leads (near $x+t=0$), to
\begin{equ}
  X=\frac{1}{4}\int_0^t \d\tau  \int_{x-t+\tau }^{x+t-\tau } \d \xi\, g(\xi+\tau)^2~.
\end{equ}
The derivatives are
\begin{equa}
  \partial_x X &= \frac{1}{4}\int_0^t \d\tau  g(x+t-\tau
  +\tau)^2-g(x-t+\tau +\tau)^2\\
   &= \frac{1}{4}\int_0^t \d\tau  g(x+t)^2-g(x-t +2\tau)^2~,\\
  \partial_x^2 X &= \frac{1}{4}\int_0^t \d\tau \,
  \partial_x(g(x+t)^2)-\partial_x(g(x-t+2\tau)^2)\\
  &=-\frac{1}{4}\left(g(x+t)^2-g(x-t)^2\right)~.
\end{equa}
Therefore, we find, in perturbation theory, the equation for $z(t)$ as
\begin{equ}\label{eq:perturb}
 \frac{1}{2} g'(z(t)+t)-\frac{\alpha}{4}(g(z(t)+t))^2=0~.
\end{equ}
Since $g(0)=-1$, $g'(0)=0$, and $g''(0)>0$, we can write
$g(x)=-1+Ax^2+\OO(x^3)$, with $A>0$.
Then \eref{eq:perturb} with $z(t)+t=\epsilon $ leads to:
\begin{equa}
  g'(\epsilon )/2=\alpha g(\epsilon )^2/4~,
\end{equa}
and therefore
\begin{equa}
  z(t)=-t+\epsilon =-t+\frac{\alpha }{4A}=-t+\frac{\alpha g(0)^2}{4g''(0)}
\end{equa}
Note that under our assumptions on $g$ the shift ${\alpha
  g(0)^2}/({4g''(0)})$ is \emph{positive}, \ie the extrema move
towards the center.


\section{Disorderd thoughts}

Here I sketch (badly) what I have in mind:

At any time $t$, I want to consider the current $u(x,t)$ as a new
initial condition. But, we are not interested in the full evolution of
the solution, but only in the motion of $z(t)$ which is defined by
$u_{xx}(z(t),t)=0$. So I consider $f(x)=u(x,t)$ and $g(x)=u_t(x,t)$ as my
initial conditions at time $t$. The ``free'' evolution from this
initial condition at time $t+\epsilon $  should be (on one side only)
\begin{equa}\label{eq:version1}
  \HALF f(x+\epsilon ) +\HALF \int_{x-\epsilon }^{x+\epsilon }  g(\xi)
  \d \xi + \frac{\alpha }{2} \int_0^\epsilon\d \tau \int_{x-\epsilon
    +\tau}^{x+\epsilon -\tau }\d \xi  (u_x(\xi,t+\tau ))^2~. 
\end{equa}
I have a feeling that the ``correct'' speed in this case should be
$z'(t)$ and not $1$  so that perhaps one should have taken for
example
\begin{equa}\label{eq:version2}
  \HALF f(x+z'(t)\epsilon ) +\HALF \int_{x-z'(t)\epsilon }^{x+z'(t)\epsilon }  g(\xi)
  \d \xi +\frac{\alpha}{2}  \int_0^{z'(t)\epsilon }\d \tau \int_{x-z'(t)\epsilon
    +\tau}^{x+z'(t)\epsilon -\tau }\d \xi  (u_x(\xi,t+\tau ))^2 ~.
\end{equa}
I have somehow the idea that the form \eref{eq:version2} should
somehow make better cancellations, but I am stuck here.

Anyway, the idea is now that \eref{eq:version1} or
\eref{eq:version2}, \emph{near the root of $u_{xx}$} can be treated
again with the implicit function idea as in \eref{eq:perturb}. 



\end{document}
