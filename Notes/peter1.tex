\documentclass[12pt,a4paper]{article}
%\usepackage[allfiguresdraft]{draftfigure}
%\newcommand{\pdfextension}{pdf}
%\newcommand{\pngextension}{png}
\usepackage{cite}
\usepackage{natbib}
\usepackage{breakcites}
\usepackage[normalem]{ulem}
\usepackage{comment}
\usepackage[font=footnotesize,labelfont=bf]{caption}
 \usepackage[dvipsnames]{xcolor}
\let\rho=\varrho

\def\fref#1{Fig.~\ref{#1}}
\def\tabref#1{Table~\ref{#1}}
\def\cref#1{Condition~\ref{#1}}
\def\Cref#1{Corollary~\ref{#1}}
\def\eref#1{(\ref{#1})}
\def\sref#1{\textsection\ref{#1}}
\def\lref#1{Lemma~\ref{#1}}
\def\rref#1{Remark~\ref{#1}}
\def\tref#1{Theorem~\ref{#1}}
\def\dref#1{Definition~\ref{#1}}
\def\pref#1{Proposition~\ref{#1}}
\def\aref#1{Assumption~\ref{#1}}
\def\qref#1{Subsection~\ref{#1}}

\def\dref#1{Definition~\ref{#1}}
\def\myundefined#1{\special{ps: 1 0 0 setrgbcolor}{what is #1?}%
	\special{ps: 0 0 0 setrgbcolor}}
\newenvironment{myitem}
{\begin{itemize}
  \setlength{\itemsep}{1pt}
  \setlength{\parskip}{0pt}
  \setlength{\parsep}{0pt}}
{\end{itemize}}

\newenvironment{myenum}
{\begin{enumerate}
  \setlength{\itemsep}{1pt}
  \setlength{\parskip}{0pt}
  \setlength{\parsep}{0pt}}
{\end{enumerate}}

%\usepackage{sub_JP}
%\usepackage{fancyhdr}
%\pagestyle{fancy}
\usepackage{amsmath}
\usepackage{amsfonts}
\usepackage{graphicx}
%\usepackage{makeidx}
\usepackage{times}
\usepackage{amsthm}
\usepackage{ amssymb }
\usepackage{color}
\usepackage{mhequ}
\usepackage{dsfont}
\usepackage[scanall]{psfrag}
\usepackage[margin=2.5cm]{geometry}
\usepackage{color}
\usepackage{url}
\usepackage{lipsum}
%\usepackage{authblk}
\usepackage{subfig}
\usepackage{mhequ}

\usepackage{tikz}
\usepackage[graphics, active, tightpage]{}

%\usepackage[expansion=true]{microtype}

%\captionsetup[figure]{margin=2cm,font=footnotesize,labelfont=bf,labelsep=endash,textfont=rm}\captionsetup[subfigure]{margin=0pt}


\def\thecomma{\ifx,\thenewxt \else\ifx;\thenext \else\ifx.\thenext
	\else\ifx!\thenext \else\ifx:\thenext\else\ifx)\thenext \else \
	\fi\fi\fi\fi\fi\fi}
\def\condblank{\futurelet\thenext\thecomma}
\def\ie{{\it i.e.,}\condblank}
\def\eg{{\it e.g.,}\condblank}

\numberwithin{equation}{section}



\newtheorem*{oneshot}{Hypothesis~\ref{h:assumption}}
\newtheorem{theorem}{Theorem}[section]
\newtheorem{lemma}[theorem]{Lemma}
\newtheorem{proposition}[theorem]{Proposition}
\newtheorem{definition}[theorem]{Definition}
\newtheorem{notation}[theorem]{Notation}
\newtheorem{observation}[theorem]{Observation}
\newtheorem{assumption}[theorem]{Assumption}
\newtheorem{hypothesis}[theorem]{Hypothesis}
\newtheorem{conjecture}[theorem]{Conjecture}
\newtheorem{example}[theorem]{Example}
\newtheorem{property}[theorem]{Property}
\newtheorem{condition}[theorem]{Condition}
\newtheorem{corollary}[theorem]{Corollary}
\theoremstyle{definition} %remarks in rm
\newtheorem{remark}[theorem]{Remark}
\def\HALF{{\textstyle\frac{1}{2}}}
\def\TWOTHIRDS{{\textstyle\frac{2}{3}}}
\def\FOURTHIRDS{{\textstyle\frac{4}{3}}}
%\bibliographystyle{JPE}
%\usepackage{cite}

\usepackage{stmaryrd}
\def\scomm#1#2{\ensuremath{\bigr\llbracket #1:#2\bigr\rrbracket}}

\newcommand{\meq}[1]{\ensuremath{\stackrel{\scriptscriptstyle#1}{\scriptstyle
			\sim}}}


\newcommand{\dd}{\mathrm{d}}
\newcommand{\dt}{\,\dd t}
\newcommand{\mpart}[2]{\frac{\partial #1}{\partial #2}}
\newcommand{\avg}[1]{\left\langle #1\right\rangle}

\newcommand{\bigoh}[1]{\hat {\mathcal O}(p_2^{#1})}
\newcommand{\bigohneg}[1]{\hat {\mathcal O}\!\left({p_2^{-#1}}\right)}

\newcommand{\jp }[1]{{\color{magenta}jp*****:  #1}}
\newcommand{\noe }[1]{{\color{red}noe:  #1}}
\newcommand{\christophe }[1]{{\color{red}christophe: #1}}
\def\argcdot{{\,\cdot\,}}


\newcommand{\cA}{{\ensuremath{\mathcal A}} }
\newcommand{\cF}{{\ensuremath{\mathcal F}} }
\newcommand{\cP}{{\ensuremath{\mathcal P}} }
\newcommand{\cE}{{\ensuremath{\mathcal E}} }
\newcommand{\cH}{{\ensuremath{\mathcal H}} }
\newcommand{\cC}{{\ensuremath{\mathcal C}} }
\newcommand{\cN}{{\ensuremath{\mathcal N}} }
\newcommand{\cL}{{\ensuremath{\mathcal L}} }
\newcommand{\cT}{{\ensuremath{\mathcal T}} }
\newcommand{\cD}{{\ensuremath{\mathcal D}} }
\newcommand{\cU}{{\ensuremath{\mathcal U}} }
\newcommand{\cV}{{\ensuremath{\mathcal V}} }
\newcommand{\cS}{{\ensuremath{\mathcal S}} }
\newcommand{\cY}{{\ensuremath{\mathcal Y}} }


%%%%%%%%%%%%%%%%%%%%%%%%%%%%%%%%%%%%%%%%%%%%%%%%%%%%%%%%%%%%%%%%%%%%%%%%%%%%%%
%%%%%%%%%%%% Blackboard bolds
%%%%%%%%%%%%%%%%%%%%%%%%%%%%%%%%%%%%%%%%%%%%%%%%%%%%%%%%%%%%%%%%%%%%%%%%%%%%%%

\newcommand{\bbA}{{\ensuremath{\mathbb A}} }
\newcommand{\bbB}{{\ensuremath{\mathbb B}} }
\newcommand{\bbC}{{\ensuremath{\mathbb C}} }
\newcommand{\bbD}{{\ensuremath{\mathbb D}} }
\newcommand{\bbE}{{\ensuremath{\mathbb E}} }
\newcommand{\bbF}{{\ensuremath{\mathbb F}} }
\newcommand{\bbG}{{\ensuremath{\mathbb G}} }
\newcommand{\bbH}{{\ensuremath{\mathbb H}} }
\newcommand{\bbI}{{\ensuremath{\mathbb I}} }
\newcommand{\bbJ}{{\ensuremath{\mathbb J}} }
\newcommand{\bbK}{{\ensuremath{\mathbb K}} }
\newcommand{\bbL}{{\ensuremath{\mathbb L}} }
\newcommand{\bbM}{{\ensuremath{\mathbb M}} }
\newcommand{\bbN}{{\ensuremath{\mathbb N}} }
\newcommand{\bbO}{{\ensuremath{\mathbb O}} }
\newcommand{\bbP}{{\ensuremath{\mathbb P}} }
\newcommand{\bbQ}{{\ensuremath{\mathbb Q}} }
\newcommand{\bbR}{{\ensuremath{\mathbb R}} }
\newcommand{\bbS}{{\ensuremath{\mathbb S}} }
\newcommand{\bbT}{{\ensuremath{\mathbb T}} }
\newcommand{\bbU}{{\ensuremath{\mathbb U}} }
\newcommand{\bbV}{{\ensuremath{\mathbb V}} }
\newcommand{\bbW}{{\ensuremath{\mathbb W}} }
\newcommand{\bbX}{{\ensuremath{\mathbb X}} }
\newcommand{\bbY}{{\ensuremath{\mathbb Y}} }
\newcommand{\bbZ}{{\ensuremath{\mathbb Z}} }


%%%%%%%%%%%%%%%%%%%%%%%%%%%%%%%%%%%%%%%%%%%%%%%%%%%%%%%%%%%%%%%%%%%%%%%%%%%%%%
%%%%%%%%%%%% Greek letters
%%%%%%%%%%%%%%%%%%%%%%%%%%%%%%%%%%%%%%%%%%%%%%%%%%%%%%%%%%%%%%%%%%%%%%%%%%%%%%

\newcommand{\ga}{\alpha}
\newcommand{\gb}{\beta}
\newcommand{\gga}{\gamma}            % \gg already exists...
\newcommand{\gd}{\delta}
\newcommand{\gep}{\varepsilon}       % \ge already exists...
\newcommand{\gp}{\varphi}
\newcommand{\gr}{\rho}
\newcommand{\gvr}{\varrho}
\newcommand{\gz}{\zeta}
\newcommand{\gG}{\Gamma}
\newcommand{\gP}{\Phi}
\newcommand{\gD}{\Delta}
\newcommand{\gk}{\kappa}
\newcommand{\go}{\omega}
\newcommand{\gto}{{\tilde\omega}}
\newcommand{\gO}{\Omega}
\newcommand{\gl}{\lambda}
\newcommand{\gL}{\Lambda}
\newcommand{\gs}{\sigma}
\newcommand{\gS}{\Sigma}
\newcommand{\gt}{\vartheta}
\let\kappa=\varkappa
\let\phi=\varphi
\def\CC{{\mathcal C}}
\def\Rsch{r_{\rm sch}}
\def\d{{\rm d}} 
\def\KK{{\mathcal K}}
\def\OO{{\mathcal O}}
\def\integer{{\mathbb Z}}
\def\real{{\mathbb R}}
\newcommand{\ind}{\mathbf{1}}
\def\p2t2{{\tilde p_2^{\,2}}}

\def\fhc#1{\textcolor{violet}{#1}}
\definecolor{bittersweet}{rgb}{1.0, 0.44, 0.37}
\newcommand{\fhn}[1]{{\textcolor{bittersweet}{\sf[#1]}}}
\newcommand{\edit}[2] {\textcolor{black}{\sout{#1}} {\textcolor{violet}{#2}} }
\newcommand{\HZ}[1]{\textcolor{ForestGreen}{HZ:#1}}
\def\JP#1{\textcolor{blue}{JPE:#1}}
\let\epsilon=\varepsilon
\usepackage{authblk}

%%%%%%%
%% Farbod Hassani
%%%%%%%
\newcommand{\HH}{\mathcal H}
\def\be{\begin{equation}}
\def\ee{\end{equation}}
\def\const{\text{const.}}

\begin{document}

\section{Divergence}
The following is a slight adaptation of the results of Wittwer, Pan
Shi, and Hassani.


We consider the equation $u_{tt}=(u_x)^2$ on the real line.
We start by writing the solution in the form
\begin{equ}\label{eq:peter}
  u(x,t)=f(x)+g(x)t +\int_0^t \d\tau \int_0^\tau  \d\tau'
  (u_x(x,\tau '))^2~. 
\end{equ}
This corresponds to the initial conditions
\begin{equ}
  u(x,0)=f(x)~,\quad  u_t(x,0)=g(x)~.
\end{equ}
We will consider the case that $f'(0)=g'(0)=0$, and we ask how the
solution behaves near $x=0$. Depending on the curvatures $f''(0)$ and
$g''(0)$, the second derivative $u_{xx}(x,t)$ will, or will not
  diverge at $x=0$. Of course, if the functions $f$ and $g$ have
  vanishing derivatives at some other point(s) $x_0$, the same
  discussion will apply at those points, and there can be one of these
  points where $u_{xx}(x_0,t)$ diverges before the one at $x=0$. In
  the following proposition, we will neglect this aspect.

  
\begin{proposition}\label{prop:peter}
  Assume $f'(0)=g'(0)=0$. Define
  \begin{equ}
    c=\HALF g''(0)^2-\TWOTHIRDS f''(0)^3~. 
  \end{equ}

  Then the following cases appear:\\
  (i) If $g''(0)>0$ then $u_{xx}(0,t)$ diverges in finite time $t_*$
  given by
  \begin{equ}\label{eq:div}
    t_*=\int_{f''(0)}^\infty \frac{\d b }{\sqrt{\FOURTHIRDS b^3 +2c}}~.
  \end{equ}\\
  (ii) If $g''(0)<0$ then $u_{xx}(0,t)$ will converge to $b_*$ in a
  finite time $t_*$, where
  \begin{equ}\label{eq:conv}
    \TWOTHIRDS b_*^3 =-c~,\text{ and }
    t_*=\int_{b_*}^{f''(0)} \frac{\d b}{{\sqrt{\FOURTHIRDS b^3 +2c}}}~.
  \end{equ}
  At this point in time, we will have  $u_t(0,t_*)=0$
  which corresponds to (iii) and the solution will diverge after
  another finite time (unless $u(0,t_*)=0$.
  \\
  (iii)  If  $g''(0)=0$, and $f''(0)\ne0$ then $u_{xx}(0,t)$ diverges
  in finite time $t_*$ given again by \eref{eq:div}.\\
  (iv) If $g''(0)=0$ and $f''(0)=0$ then $u_{xx}(0,t)$ stays constant.
\end{proposition}
\begin{remark}
  The divergence and limit times above are standard elliptic
  integrals. When $g''(0)=0$, the finite divergence time $t_*$ scales like $\OO(|f''(0)|^{-1/2})$.
\end{remark}
\begin{remark}
  A case of a certain interest in cosmology appears when
  $f''(0)=0$ and $g''(0)>0$.  In that case, $t_*\sim 2.547/(g''(0))^{1/6}$.
\end{remark}
\begin{proof}



Define 
\begin{equa}
  a(t)= u_x(0,t)~,\quad b(t)=u_{xx}(0,t)~.
\end{equa}
Then \eref{eq:peter} leads to
\begin{equa}
  a(t)&=f'(0)+g'(0)t+2\int_0^t \d\tau \int_0^\tau  \d\tau' u_x(0,\tau ')
  u_{xx}(0,\tau ')~,
\end{equa}
and therefore
\begin{equa}\label{eq:a}
  \ddot a(t)=2 a(t)\,b(t)~.
\end{equa}
Similarly,
\begin{equa}
  b(t)=f''(0)+g''(0)t +2\int_0^t \d\tau \int_0^\tau  \d\tau'
\left(  (u_{xx}(0,\tau '))^2
  +u_{x}(0,\tau ')u_{xxx}(0,\tau ')\right)~,
\end{equa}
From this, we deduce
\begin{equa}[eq:b]
  \ddot b(t)&=
  2(u_{xx}(0,t))^2
  +2u_{x}(0,t)u_{xxx}(0,t)\\
  &= 2b(t)^2 + 2a(t)u_{xxx}(0,t)~.
\end{equa}
Since we assume $f'(0)=g'(0)=0$ we find from \eref{eq:a} that
$a(t)=0$ for all $t$ for which $b(t)$ is finite. Therefore,
\eref{eq:b} reduces to
\begin{equa}\label{eq:bdd}
\ddot b(t)=2(b(t))^2~.  
\end{equa}
We will discuss this equation. For computing the divergence time, it
is useful to transform the equation as follows: Multiplying by $\dot b$
leads to
\begin{equa}
  \HALF \frac{\d}{\d t}(\dot b(t))^2 = \TWOTHIRDS \frac{\d }{\d
  t}b(t)^3~,
\end{equa}
or, for some $c$,
\begin{equa}\label{energy}
  \HALF(\dot b(t))^2  = \TWOTHIRDS(b(t))^3 +c~,
\end{equa}
Note that looking at $t=0$ we find
\begin{equ}\label{eq:c}
  c=\HALF \dot b(0)^2 -\TWOTHIRDS b(0)^3= \HALF  g''(0)^2 -\TWOTHIRDS f''(0)^3~,
\end{equ}
which is the definition in the proposition.



We consider first the case where
$g''(0)>0$. Then $\dot b(0)>0$ and from \eref{energy} we find that
\begin{equa}\label{eq:dotb}
  \dot b(t)  =\sqrt{\FOURTHIRDS (b(t))^3 +2c}~,
  \end{equa}
which means $b$ is increasing. Using standard techniques, we get 
\begin{equa}
  \d t  = \frac{\d b}{\sqrt{\FOURTHIRDS b^3 +2c}}~.
\end{equa}
From \eref{eq:dotb} we deduce 
the divergence time $t_*$,
\begin{equation}\label{eq:t2}
  t_*=\int_{b(0)}^\infty \frac{\d b }{\sqrt{\FOURTHIRDS b^3 +2c}}~.
\end{equation}
This proves \eref{eq:div}.

The case $g''(0)<0$ is handled similarly, but now \eref{eq:dotb} is
replaced by 
\begin{equa}\label{eq:dotbminus}
  \dot b(t)  =-\sqrt{\FOURTHIRDS (b(t))^3 +2c}~,
\end{equa}
This means that $b$ is decreasing until the square root in
\eref{eq:dotbminus} vanishes. This defines $b_*$, and then \eref{eq:t2} is replaced by
\begin{equ}
  t_*=\int_{b_*}^{f''(0)} \frac{\d b }{\sqrt{\FOURTHIRDS b^3 +2c}}~.
\end{equ}
This leads to \eref{eq:conv}.

The assertions under (iii) are a simple variant of (i) and (ii). The
difference is that because $g''(0)=0$, we find now that $c=-\TWOTHIRDS
f''(0)^3$, and $c\ne0$ by the assumption $f''(0)\ne0$.
The only remaining case is (iv), $g''(0)=f''(0)=0$, which directly
leads to $b(t)=b(0)$ by \eref{eq:bdd}.
\end{proof}

\end{document}
